\documentclass[12pt]{article}
\usepackage{graphics}

\def\R{\hbox{\rm I\kern-.2150em R}}
\def\B{\hbox{\rm I\kern-.2150em B}}
\def\maps{:}
\def\into{\rightarrow}
\def\QED{\vrule height 6 pt width 5 pt depth 0pt}
\def\nopageref#1{}
\def\SYNTAX{2.5in}
\def\MSG{2.5in}

\newtheorem{Definition}{Definition}
\newtheorem{Lemma}{Lemma}
\newtheorem{Task}[Definition]{Task}
\newtheorem{Corollary}[Lemma]{Corollary}
\newtheorem{Conjecture}[Lemma]{Conjecture}

\title{NLPAPI: An API to Nonlinear Programming Problems. {\bf User's Guide}}
\author{
  Michael E. Henderson\\[.5\baselineskip]
  IBM Research Division\\
  T. J. Watson Research Center\\
  Yorktown Heights, NY ~10598\\
  {\tt mhender@watson.ibm.com}
 }

\begin{document}

  \maketitle

   \section{Introduction}

    This API provides a way to create and access Nonlinear Programming Problems of the
    form 
    \begin{displaymath}
     \begin{array}{ll}
      {\rm minimize}~O({\bf v})& \qquad\qquad \rm Objective~Function\\
      {\rm subject~to:}\\
      \qquad l_i \leq v_i \leq u_i& \qquad\qquad  \rm Simple~Bounds\\
      \qquad c_i({\bf v}) = 0& \qquad\qquad \rm Nonlinear~Equality
                                                      ~Constraints\\
      \qquad L_i \leq c_i({\bf v}) \leq U_i& \qquad\qquad \rm Nonlinear
                                                      ~Inequality~Constraints\\
     \end{array}
    \end{displaymath}

     The API began as an interface to the LANCELOT optimization code.
    LANCELOT is a Fortran program that solves nonlinear optimization 
    problems using a trust region method. It has its own input description language called SIF, 
    which is run through a program called the ``SIF Decoder'', which 
    produces a set of Fortran subroutines and data files. 
    These are compiled and linked against a supplied main program,
    which reads the data files and calls the subroutines
    in order to solve the problem, then writes the solution to a file.
    Parameters controlling the execution are read from another file.

    We wanted to use LANCELOT on a problem whose constraints involved functions
    that were computed by an external code (a circuit simulator), and so could
    not be expressed in the SIF language. There is a hook to allow "external"
    functions in SIF, and this worked, but was awkward. I designed this API to
    replace the SIF decoder. It allows the user to build up a problem
    with a sequence of subroutine calls, to which the user may pass pointers 
    to those external routines. The user may also set LANCELOT's parameters, and
    then invoke LANCELOT to solve the problem.

    LANCELOT is documented in the book ``LANCELOT: a Fortran Package for
    Large-Scale Nonlinear Optimization (Release A)'', by A. R. Conn, 
    N. I. M. Gould and Ph. L. Toint, {\it Springer Series in Computational 
    Mathematics, Volume 17, Springer Verlag (Heidelberg, New York), 
    ISBN 3-540-55470-X, 1992.} This document only covers the API.

    The project has grown somewhat, into an API which can be used to define a
    nonlinear program that might be presented to any solver.
    In fact, we have so far only used LANCELOT and IPOPT, but a CUTE interface
    is in the works, and that will allow a number of solvers to be used.

    The CUTE environment is documented in the paper ``CUTE: Constrained and Unconstrained
    Testing Environment'', by I. Bongartz, A. R. Conn, Nick Gould, and Ph. L. Toint,
    ACM Transactions on Mathematical Software, Vol. 21, No. 1, March 1995, Pages 123--160.
    A web page on a follow-on called "CUTEr" is available at {\tt http://hsl.rl.ac.uk/cuter-www/}.

   \section{Defining a Problem}

   The API deals with minimization (or maximization) problems of the form
   \begin{displaymath}
     \begin{array}{ll}
      {\rm minimize}~O({\bf v})& \qquad\qquad \rm Objective~Function\\
      {\rm subject~to:}\\
      \qquad l_i \leq v_i \leq u_i& \qquad\qquad  \rm Simple~Bounds\\
      \qquad c_i({\bf v}) = 0& \qquad\qquad \rm Nonlinear~Equality
                                                      ~Constraints\\
      \qquad L_i \leq c_i({\bf v}) \leq U_i& \qquad\qquad \rm Nonlinear
                                                      ~Inequality~Constraints\\
     \end{array}
   \end{displaymath}

     To define the problem the user creates an {\tt NLProblem} that
    holds all of the information defining the problem. The problem is initially
    empty, and the user must set the objective, and add equality and inequality constraints.
    This can be done either by specifying an expression, supplying subroutines to evaluate the
    objective/constraints, or building them as a list of groups (i.e. LANCELOT's form). For example
    \begin{displaymath}
     O({\bf v}) = \sum_{i=1}^{ng} {1\over s_i}g_i( \sum_{j=1}^{ne_i} w_{ij} f_{ij}(R {\bf e}_{ij}) 
             + <a_i,{\bf v}> - b_i)
    \end{displaymath}
    The $g_i:\R\into\R$ are caled groups functions, the $f_{ij}:\R^{e_{ij}}\into\R$ are called nonlinear
    elements, $<a_i,{\bf v}>$ is called the linear element of the group and $b_i$ is called 
    the constant element.

     To invoke a solver the user creates a solver, e.g. an {\tt NLLancelot} data structure.
    This has a set of parameters and a routines for invoking the solver.
    Default values are assigned to the parameters controlling the solver, and the user 
    can set parameters as needed.

    Below we describe the "C" interface. A discussion of how to call the routines from Fortran 
    is deferred to a later section.

   \subsection{The Nonlinear Optimization Problem}

     To begin the user creates a problem --
     \begin{verbatim}
       #include <NLPAPI.h>

       NLProblem P;
       P=NLCreateProblem("MyProblem",1004);
     \end{verbatim}
     The first argument is a name that is assigned to the problem. The second is the number of variables in 
     the problem.

     Each variable has a name, and may have simple bounds. The default for variable names is {\tt "X\%d"}. So in the
     example, the default is {\tt "X1"}, ... {\tt "X1004"}. The names can be changed, and simple bounds imposed with
     the routines:
     \begin{verbatim}
       int i;
       double u,l;

       rc=NLPSetVariableName(P,550,"V");
       rc=NLPSetSimpleBounds(P,i,l,u);
       rc=NLPSetLowerSimpleBound(P,i,l);
       rc=NLPSetUpperSimpleBound(P,i,u);
     \end{verbatim}

   \subsubsection{The Objective}

     Each problem has an objective function. This can be set with the {\tt NLPSet\-Objective} or
     {\tt NLPSet\-Objective\-ByString} routine, or can be built as the sum of a number of groups.
     \begin{verbatim}
      NLProblem P;
      char name[]="Obj";
      int nv;
      int *v;
      double (*F)(int,double*,void*);
      double (*dF)(int,int,double*,void*);
      double (*ddF)(int,int,int,double*,void*);
      void *data;
      void (*freedata)(void*);

      nv=3;v[0]=0;v[1]=45;v[2]=9;
      rc=NLPSetObjective(P,name,nv,v,F,dF,ddF,data,freedata);
     \end{verbatim}

     The subroutines {\tt F}, {\tt dF} and {\tt ddF} are called back when the objective is evaluated. The
     arguments to {\tt F} are:
     \begin{verbatim}
   double F(int nv, double *x, void *data)
      int nv     The number of entries in x, as provided by
                        the user in the SetObjective call.
      double *x  An array with the values of the coordinates
                        of x.
      void *data The data pointer provided by the user in the
                        SetObjective call.
     \end{verbatim}
     The {\tt data} variable allows the user to associate a data block with the objective, that is passed to the 
     functions when they are evaluated. The {\tt freedata} routine is called when the problem is free'd, so that
     the data block can be released. The array {\tt v} lists the {\tt nv} problem variables on which the objective
     (or constraint, since the same form is used for those) depends.This allows
     a simple form of sparsity).
      
     {\tt dF} evaluates the partial derivatives of {\tt f}, and has an
     additional integer argument (the first argument), which
     indicates which partial derivative to return. {\tt ddF} evaluates the
     second partial derivatives, and has two additional integer arguments
     (the first two).

     Alternatively, the user can define the objective by means of a string containing an expression:
     \begin{verbatim}
        NLPSetObjectiveByString(P,name,nv,v,
            "[x1,x2,x3]",
            "(x1-x2)**2+(x1+x2-10)**2/9+(x3-5)**2");
     \end{verbatim}
     In this case the array {\tt v} lists the {\tt nv} variables whose values are substituted for the variables 
     {\tt "[x1,x2,x3]"}. So in the last string {\tt x1} is the value of the first problem variable. {\tt x1} is the
     value of the 46th problem variable, and so on.

     Finally, the objective may be defined via LANCELOT's Group Partial Separable form. When the problem is 
     first created the objective has a single group with trivial group function, no nonlinearelements, and
     linear and constant elements both zero. Additional groups ban be added using the
     {\tt NLPAdd\-Group\-To\-Objective} routine. Each group has a group function $g_i$, a group scale $s_i$,
     a linear element, ${\bf a}_i.{\bf v}$, a constant element $b_i$, and a nonlinear element $N_i({\bf v})$.
     (See the sections below on ``Groups''). 
     \begin{displaymath}
       O({\bf v}) = \sum {1\over{s_i}}g_i(N_i({\bf v})+ {\bf a}_i.{\bf v}-b_i)
     \end{displaymath}
     The various pieces are added or set with the routines
     \begin{verbatim}
       NLPSetObjectiveGroupFunction
       NLPSetObjectiveGroupScale
       NLPSetObjectiveGroupA
       NLPSetObjectiveGroupB
       NLPAddNonlinearElementToObjectiveGroup
     \end{verbatim}
     The nonlinear element is itself composed of a sum of element functions:
     \begin{displaymath}
       N({\bf v}) = \sum w_i f_i( \sum R_{ij} e_j )
     \end{displaymath}
     for a more complete description see the section below on nonlinear elements.

     The main advantage of this form is that the derivatives of the objective (and other functions) can be
     expressed in terms of the derivatives of simpler functions. This is a convenience, but if the derivatives
     are approximated by differencing it can also substantially reduce the number of operations needed to approximate
     the derivatives of the objective.

     The objective can be evaluated using the routines:
     \begin{verbatim}
      double o;
      NLVector v,g;
      NLMatrix H;

      o=NLPEvaluateObjective(P,v);

      g=NLCreate...Vector(...);
      NLPEvaluateGradientOfObjective(P,v,g);

      H=NLCreate...Matrix(...);
      NLPEvaluateHessianOfObjective(P,v,H);
     \end{verbatim}
     The idea here is that the user creates either a sparse or dense vector, and passes it to the routine
     which computes the gradient, which fills in the appropriate values, or a matrix in one of several 
     formats, and passes it to the routine which evaluates the Hessian. (See the sections below on vectors
     and matrices.)

   \subsubsection{Inequality Constraints}

     A problem may have a number of inequality constraints. They are handled almost exactly as the objective was handled,
     but have in addition upper and lower bounds. Inequality constraints are added with the {\tt NLPAdd\-Inequality\-Constraint}
     or {\tt NLPAdd\-Inequality\-Constraint\-ByString} routine, or can be built as the sum of a number of groups.
     \begin{verbatim}
      NLProblem P;
      char name[]="InEq0";
      double l,u;
      int nv;
      int *v;
      double (*F)(int,double*,void*);
      double (*dF)(int,int,double*,void*);
      double (*ddF)(int,int,int,double*,void*);
      void *data;
      void (*freedata)(void*);

      nv=3;v[0]=3;v[1]=10;v[2]=9;
      l=1.;u=10.;
      rc=NLPAddInequalityConstraint(P,name,l,u,nv,v,F,dF,ddF,
                                              data,freedata);
     \end{verbatim}
     The {\tt data} variable allows the user to associate a data block with the objective, that is passed to the
     functions when they are evaluated. The {\tt freedata} routine is called when the problem is free'd, so that
     the data block can be released. The array {\tt v} lists the {\tt nv} variables on which the function depends.

     Alternatively, the user can define the objective by means of a string containing an expression:
     \begin{verbatim}
        rc=NLPAddInequalityConstraintByString(P,name,nv,v,l,u,
            "[x1,x2,x3]","48-x1**2-x2**2-x3**2");
     \end{verbatim}
     The array {\tt v} lists the {\tt nv} variables whose values are substituted for the variables
     {\tt "[x1,x2,x3]"}.

     When the constraint is first created it has a single, empty group. Additional groups can be added using the
     {\tt NLPAdd\-Group\-To\-Inequality\-Constraint} routine. Each group has a group function, a scale, a linear element,
     a constant element, and a set of nonlinear elements. (See the section below on ``Groups'').
     \begin{verbatim}
       NLPAddNonlinearInequalityConstraint(P,name);

       NLPAddLinearInequalityConstraint(P,name,a,b);

       NLPSetInequalityConstraintBounds
       NLPSetInequalityConstraintUpperBound
       NLPSetInequalityConstraintLowerBound

       NLPSetInequalityConstraintGroupFunction
       NLPSetInequalityConstraintGroupScale
       NLPSetInequalityConstraintGroupA
       NLPSetInequalityConstraintGroupB
       NLPAddNonlinearElementToInequalityConstraintGroup
     \end{verbatim}

     Inequality constraints can be evaluated using the routines:
     \begin{verbatim}
      int c;
      double o;
      NLVector v,g;
      NLMatrix H;

      o=NLPEvaluateInequalityConstraint(P,c,v);

      g=NLCreate...Vector(...);
      NLPEvaluateGradientOfInequalityConstraint(P,c,v,g);

      H=NLCreate...Matrix(...);
      NLPEvaluateHessianOfInequalityConstraint(P,c,v,H);
     \end{verbatim}

   \subsubsection{Equality Constraints}
     Equality constraints are handled exactly as the inequality constraints are handled, but without the
     upper and lower bounds. Equality constraints are added with the {\tt NLPAdd\-Equality\-Constraint}
     or {\tt NLPAdd\-Equality\-Constraint\-ByString} routine, or can be built as the sum of a number of groups.
     \begin{verbatim}
      NLProblem P;
      char name[]="Eq0";
      double l,u;
      int nv;
      int *v;
      double (*F)(int,double*,void*);
      double (*dF)(int,int,double*,void*);
      double (*ddF)(int,int,int,double*,void*);
      void *data;
      void (*freedata)(void*);

      nv=3;v[0]=3;v[1]=10;v[2]=9;
      l=1.;u=10.;
      rc=NLPAddEqualityConstraint(P,name,l,u,nv,v,F,dF,ddF,
                                              data,freedata);

     \end{verbatim}
     The {\tt data} variable allows the user to associate a data block with the objective, that is passed to the
     functions when they are evaluated. The {\tt freedata} routine is called when the problem is free'd, so that
     the data block can be released. The array {\tt v} lists the {\tt nv} variables on which the function depends.

     Alternatively, the user can define the objective by means of a string containing an expression:
     \begin{verbatim}
        rc=NLPAddEqualityConstraintByString(P,name,nv,v,l,u,
            "[x1,x2,x3]",
            "48-x1**2-x2**2-x3**2");
     \end{verbatim}
     Again, the array {\tt v} lists the {\tt nv} variables whose values are substituted for the variables
     {\tt "[x1,x2,x3]"}.

     When the constraint is first created it has a single, empty group. Additional groups are added using the
     {\tt NLPAdd\-Group\-To\-Equality\-Constraint} routine. Each group has a group function, a scale, a linear element,
     a constant element, and a set of nonlinear elements. (See the section below on ``Groups'').
     \begin{verbatim}
       NLPAddNonlinearEqualityConstraint(P,name);

       NLPAddLinearEqualityConstraint(P,name,a,b);

       NLPSetEqualityConstraintGroupFunction
       NLPSetEqualityConstraintGroupScale
       NLPSetEqualityConstraintGroupA
       NLPSetEqualityConstraintGroupB
       NLPAddNonlinearElementToEqualityConstraintGroup
     \end{verbatim}

     Equality constraints can be evaluated using the routines:
     \begin{verbatim}
      int c;
      double o;
      NLVector v,g;
      NLMatrix H;

      o=NLPEvaluateEqualityConstraint(P,c,v);

      g=NLCreate...Vector(...);
      NLPEvaluateGradientOfEqualityConstraint(P,c,v,g);

      H=NLCreate...Matrix(...);
      NLPEvaluateHessianOfEqualityConstraint(P,c,v,H);
     \end{verbatim}

   \subsubsection{Transformations of the Problem}

    Several common operations on problems are provided. They are not necessary, but may be of help. A sophisticated 
    user may of course write their own.

    The first operation simply creates a copy of the problem:
    \begin{verbatim}
     NLProblem Q;

     Q=NLCopyProblem(P);
    \end{verbatim}
    The other transformations below change the problem, so a copy can be useful to compare results.

    Next there is a transformation which looks for simple bounds where the upper and lower bounds are
    identical, and adds a linear equality constraint which requires that the variable take that value, 
    and removes the simple bounds.
    \begin{verbatim}
     NLEliminateFixedVariables(P);
    \end{verbatim}
    This is useful for interior point techniques, which replace simple bounds by log barriers, and have
    problems dealing with identical bounds.

    Inequalities are sometimes dealt with by introducing extra variables called slacks. That is,
    \begin{displaymath}
      l\leq f({\bf v})\leq u
    \end{displaymath}
    is replaced by an equality constraint and simple bounds on the slack --
    \begin{displaymath}
    \begin{array}{l}
      \displaystyle f({\bf v})-s =0 \\
      \displaystyle 0\leq s \leq u-l\\
    \end{array}
    \end{displaymath}
    These operations take a problem with inequality and equality constraints and convert it to a problem
    with only equality constraints --
    \begin{verbatim}
      NLPConvertToEqualityAndBoundsOnly(P);
    \end{verbatim}

    Finally, equalities are sometimes eliminated by introducing a quadratic penalty and Lagrange multipliers.
    That is, 
    \begin{displaymath}
    \begin{array}{lr}
      \displaystyle {\rm minimize}&\displaystyle\quad O({\bf v}) \\
      \displaystyle {\rm subject to}& \displaystyle\quad f({\bf v})=0\\
    \end{array}
    \end{displaymath}
    becomes
    \begin{displaymath}
      {\rm minimize}\quad O({\bf v}) + {1\over{2\mu}}f^2({\bf v})-\lambda f({\bf v})
    \end{displaymath}
    In the limit of small penalty parameter $\mu$, and minimizing w.r.t. both ${\bf v}$ and $\lambda$, the
    solution of this problem is the same as the solution of the original problem.

    This is a little complicated, because of the group structure. In fact, the terms added to the objective are
    \begin{displaymath}
      {\rm minimize}\quad O({\bf v}) + {1\over{2\mu}}\sum_i \left( f_i({\bf v})-\mu\lambda_i\right)^2
    \end{displaymath}
    Notice that this perserves the group structure (see the floowing sections) {\it if} the constraint has
    only one group and the trivial group function. This is required to use this transformation. If this is
    true the groups added to the objective get a group function $g(x)=x^2$,
    inherit the nonlinear and linear elements of the constraint. The constant element of the group
    is increased by $\mu\lambda_i$, and the group scale is squared and then multiplied by $2\mu$. This
    means that when the penalty parameter $\mu$ or the Lagrange multipliers ($\lambda_i$'s) are changed 
    we must know which groups in the objective are penalties, and or each, what the original constant
    element and group scale were. Therefore the routine which creates the terms in the objective 
    requires arrays that it can store this information in --
    \begin{verbatim}
      int    g[nc];  /* the ids of the added groups */
      double mu;
      double l[nc];  /* Lagrange multipliers */
      double b[nc];  /* Constant elements */
      double s[nc];  /* Group scales */

      nc=NLPGetNumberOfEqualityConstraints(P);

      NLCreateAugmentedLagrangian(P, mu,l, g,b,s);

      NLSetLambaAndMuInAugmentedLagrangian(P, nc, mu,l, g,b,s);
    \end{verbatim}

   \subsubsection{Groups}
     In LANCELOT, functions are made of a sum of {\it groups}. Each group is a scaled scalar function of a nonlinear
     function of the problem variables. A group is of the form
     \begin{displaymath}
      \sum {1\over{s_i}}g_i(N_i({\bf v})+ {\bf a}_i.{\bf v}-b_i)
     \end{displaymath}
     When a group is created the group function $g_i$ is the identity, and the group scale is 1. In addition, there are no
     nonlinear elements (the $N_i({\bf v})$), and the linear element ${\bf a}_i.{\bf v}$ and the constant element
     $b_i$ are zero. 

     The user creates a group when he adds a constraint, or when he adds a group to the objective or constraint.
     The groups are associated with a constraint or the objective, so the user sets, e.g. group 3 in the objective,
     or group 2 in equality constraint 10, and so on. The group scale, linear and constant elements are set with routines 
     \begin{verbatim}
       int g;
       int c;
       NLGroupFunction gf;
       double s;
       NLVector a;
       double b;

       NLPSetObjectiveGroupFunction(g,gf);
       NLPSetEqualityConstraintGroupFunction(c,g,gf);
       NLPSetInequalityConstraintGroupFunction(c,g,gf);

       NLPSetObjectiveGroupScale(g,s);
       NLPSetEqualityConstraintGroupScale(c,g,s);
       NLPSetInequalityConstraintGroupScale(c,g,s);

       NLPSetObjectiveGroupA(g,a);
       NLPSetEqualityConstraintGroupA(c,g,a);
       NLPSetInequalityConstraintGroupA(c,g,a);

       NLPSetObjectiveGroupB(g,b);
       NLPSetEqualityConstraintGroupB(c,g,b);
       NLPSetInequalityConstraintGroupB(c,g,b);
     \end{verbatim}
     The NLVector data structure is described below (dense or spares vectors). The NLGroupFunction data structure
     represents a scalar function (with first and second derivatives). It can be created by passing routines, or
     by way of a string.
     \begin{verbatim}
       NLGroupFunction g;
       NLProblem P;
       double (*G)(double,void*);
       double (*dG)(double,void*);
       double (*ddG)(double,void*);
       void *data;
       void (*freedata)(void*);

       gf=NLCreateGroupFunction(P,"type",G,dG,ddG,
                                         data,freedata);
     \end{verbatim}
      The type is a string associated with the group. G,dG, and ddG are functions which evaluate the group function,
      and its first and second derivatives. If dG and/or ddG is NULL centered differencing is used. The data is a
      block of memory passed to the functions (so that the same G etc. can be used in different GroupFunctions), and
      freedata is a routine that is called when the GroupFunction is freed.

      A second method of creating a group function is "ByString". For example:
     \begin{verbatim}
       g=NLCreateGroupFunctionByString(P,"type",
                                         "s","sin(s)*cos(2*s)");
     \end{verbatim}

      Each "CreateGroupFunction" should be matched with a "NLFreeGroupFunction" later on in the users code. Reference
      counting is used, so a group function that is passed to Set...GroupFunction can be safely "Free'd" immediately 
      afterward.

   \subsubsection{Nonlinear Elements}

      Nonlinear elements are scalar valued functions of a subset of the problem variables. A group has a list of nonlinear
      elements whose values are summed, then added to the value of the linear and constant elements to give the argument
      to the group function. Each nonlinear element is of the form:
     \begin{displaymath}
          N({\bf v}) = \sum_i w_i f_i( \sum_j R_{ij} e_j )
     \end{displaymath}
     The element weight $w_i$ is a scalar (default is $w_i=1$). The element function $f_i$ is a scalar valued function
     of a set of {\it internal variables}, which are a linear combination of the element variables $e_j$ (a subset of
     the problem variables). The range transformation $R_{ij}$ relates element variables to internal variables.

     An element function is created with one of the routines:
     \begin{verbatim}
      NLElementFunction ef;
      NLMatrix R;
      double (*F)(int,double*,void*);
      double (*dF)(int,int,double*,void*);
      double (*ddF)(int,int,int,double*,void*);
      void *data;
      void (*freedata)(void*);
      NLMatrix ddF0;

      ef=NLCreateElementFunction(P,"etype",n,R,F,dF,ddF,
                                             data,freedata);

      ef=NLCreateElementFunctionWithInitialHessian(P,"etype",
                                             n,R,F,dF,ddF,
                                             data,freedata,
                                             ddF0);

      ef=NLCreateElementFunctionByString(P,"etype",n,R,
                                             "[x,y,z,w]",
                                             "x**2+y**2-z*w");
     \end{verbatim}
     Here, {\tt n} is the number of element variables, {\tt R} the range transformation (or NULL), F, dF, and ddF
     are routines which evalute F and its derivatives (ddF may be NULL). If ddF is NULL, ddF0 gives an initial
     guess at the Hessian which is then updated using rank one updates. Note that the updates are done on the
     derivatives w.r.t. the internal variables, and that the derivatives are derivatives w.r.t the internal variables.

     NLMatrices, which are used to represent the range transformation and the initial Hessian are described below.

     A nonlinear element is created with the routine:
     \begin{verbatim}
        NLNonlinearElement N;
        NLElementFunction ef;
        int *vars;

        N=NLCreateNonlinearElement(P,"type",ef,vars);
     \end{verbatim}
     the {\tt vars} array gives a list of the problem variables (by number, starting with 0!) which become the
     element variables.

     A nonlinear element can be added to a group using the appropriate routine:
     \begin{verbatim}
     int c;
     int g;
     double w;
     NLNonlinearElement N;

     NLPAddNonlinearElementToObjectiveGroup(P,g,w,N):
     NLPAddNonlinearElementToEqualityConstraintGroup(P,c,g,w,N):
     NLPAddNonlinearElementToInequalityConstraintGroup(P,c,g,w,N):
     \end{verbatim}
     where of course, w is the element weight.

     Each "NLCreateElementFunction..." and "NLCreateNonlinearElement..." should be paired with a "NLFreeElementFunction" 
     and "NLFreeNonlinearElement" (see Memory Management below).

   \subsubsection{Vectors}

   The NLVector is a data structure for ... vectors! There are two ``types'' of vector currently supported: sparse and
   dense. Sparse vectors are stored as a list of non-zero coordinates, dense vectors as a contiguous array of coordinates.
   They are created using the routines:
   \begin{verbatim}
    NLVector NLCreateVector(int n);
    NLVector NLCreateVectorWithSparseData(int n,int nz,
                                          int *el,double *vl);

    NLVector NLCreateDenseVector(int n);
    NLVector NLCreateVectorWithFullData(int n,double *vl);
    NLVector NLCreateDenseWrappedVector(int n,double *data);
   \end{verbatim}
   The first two create sparse vectors (n is the dimension of the vector, nz the number of nonzeroes, and the
   coordinate el[i] is given by vl[i]). The third routine creates a vector whose coordinates are all zero.
   The fourth routine creates a dense vector, and the coordinates in vl are copied into a new array. The dense
   wrapped vector stores a pointer to the data array. This allows the user to change the vector by changing the
   data array.

   Access to the vector is provided through routines like:
   \begin{verbatim}
     c=NLVGetC(v,i);
     NLVSetC(v,i,c);
   \end{verbatim}
   these have high overhead, so I'd recommend instead that you use a wrapped dense vector, or do your manipulations
   before creating the vector. Internal routines access the data directly when they can, so avoid the overhead.

   \subsubsection{Matrices}

    NLMatrices are similar to the NLVectors. There are dense matrices, and two kinds of sparse matrices currently supported,
   \begin{verbatim}
     int n,m;
     double *data;

     NLMatrix NLCreateMatrix(n,m);
     NLMatrix NLCreateMatrixWithData(n,m,data);
     NLMatrix NLCreateDenseWrappedMatrix(n,m,data);

     NLMatrix NLCreateSparseMatrix(n,m);
     NLMatrix NLCreateWSMPSparseMatrix(n);
   \end{verbatim}
   The first three constructors are for dense matrices (i.e. range transformations). The first creates and $n\times m$
   matrix with zero elements. The send copies the array data into the matrix, and the third uses a pointer to the
   data array (so that changing data changes the elements of the array). The dense matrices are stored by column,
   \`a la FORTRAN, so that element $i,j$ is located in entry $data[i+n*j]$.

   The first sparse format ({\tt NLCreateSparseMatrix}) stores a list of elements, together with the associated
   row and column. The second sparse format stores the matrix as sparse rows. 
   Access to the vector is provided through routines like:
   \begin{verbatim}
    NLMatrix A;
    Aij=NLMGetElement(A,i,j);
    NLMSetElement(A,i,j,Aij);
   \end{verbatim}
   for sparse formats setting an element creates a nonzero element (if Aij is nonzero).

   \subsubsection{Memory Management}

   When the problem is no longer needed 
   \begin{verbatim}
    NLFreeProblem(P);
   \end{verbatim}
   releases the storage. It calls {\tt NLFree..} for all of the groups, element functions, and so on which are stored
   in the problem. When the user creates one of these data structures a "reference count"  associated  with it
   is set to "1". When the problem stores a pointer to the data structure the reference count is increased by one. The
   "NLFree..." routines decreases the reference count by one and if the count is zero, releases the memory used by the
   data structure. For example:
   \begin{verbatim}
      g=NLCreateGroupFunction(...);      ref count = 1
      NLPSetObjectiveGroupFunction(...); ref count = 2
      NLFreeGroupFunction(...);          ref count = 1 not yet

      NLFreeProblem(...);                ref count = 0 DELETE g!
   \end{verbatim}
   This ensures that memory is released, but not until everyone who is using it is done with it. It relies on
   an ``honor code''. If you decided to free the group twice in the code segment above you could get some nice
   side effects.
   
   \subsubsection{Error Handling}

   Most routines return a return code that indicates whether the operation was successful. If the routine creates
   or returns a data structure an invalid value is returned if the routine is not successful. In addition a simple
   error handling is also provided. 
   \begin{verbatim}
    int NLGetNErrors();
   \end{verbatim}
   returns the total number of errors that have occured, and
   \begin{verbatim}
    void NLClearErrors();
   \end{verbatim}
   resets the count. Individual errors can be examined with the routines
   \begin{verbatim}
    int NLGetErrorSev(int n);
    char *NLGetErrorRoutine(int n);
    char *NLGetErrorMsg(int n);
    int NLGetErrorLine(int n);
    char *NLGetErrorFile(int n);
   \end{verbatim}
   The severity is 4, 8 or 12, the Routine is the routine which issued the error, and the line and file give the
   line of source code where it was issued. The message usually gives information about what caused the
   error (usually an invalid argument to the routine).

   \section{Solving a Nonlinear Optimization Problem Solver with LANCELOT}

    The LANCELOT Nonlinear Optimization Problem Solver consists of a 
    parameter list. The problem solver is created by a subroutine call,
    and has a set of default parameters, which can be modified. The
    user invokes LANCELOT by passing a problem (see above) and a 
    starting guess. The same problem solver may be used on different
    problems, and several problem solver may be created. 
    \begin{verbatim}

     NLLancelot L=NLCreateLancelot();
    \end{verbatim}
    Various parameters, like how much to print to the screen, are given
    defaults that can be queried or set via additional subroutine calls.
    For example
    \begin{verbatim}
     void LNSetPrintLevel(Lancelot,int);
    \end{verbatim}
    gives the value for the {\tt PRINT-LEVEL} line in the {\tt SPEC.SPC} 
    file.
   
    When the "LNMinimize" or "LNMaximize" routine is called, a {\tt SPEC.SPC}
    file containing the current parameters is dumped, as well as an 
    {\tt OUTSDIF.d} file containing information about the problem. A global
    variable is set to point to the problem being solved, and the Lancelot
    main program is invoked. This calls back to {\tt ELFUNS}, {\tt GROUPS},
    etc., which refer to the problem referenced by the global pointer
    to provide information to Lancelot.  When Lancelot terminates the 
    {\tt SOLUTION.d} file is read to get the solution, which is sent back 
    to the user.
    \begin{verbatim}
        LNMinimize(L,P,v0,v);
    \end{verbatim}

\section{Example}

  We will develop the code for creating and solving HS65. HS65 is the problem:
  \begin{displaymath}
  \begin{array}{l}
    {\rm minimize}\\
    \qquad {(x_1-x_2)}^2 + {(x_1+x_2-10)}^2/9 +{(x_3-5)}^2\\
    {\rm subject~to}\\
    \qquad -4.5 \leq x_1 \leq 4.5\\
    \qquad -4.5 \leq x_2 \leq 4.5\\
    \qquad -5. \leq x_3 \leq 5.\\
    \qquad 48-x_1^2 -x_2^2 -x_3^2\geq 0\\
   \end{array}
  \end{displaymath}

  First we do this using the SetObjective/AddConstraint routines, then in  
  group partially separable form.

  \subsection{Using the SetObjective/AddConstraint routines}

   This should be fairly clear. First we include the NLPAPI header file and declare some variables --
   \begin{verbatim}
   #include <NLPAPI.h>
   
   int main(int argc, char *argv[])
    {
     NLProblem P;
     int v[3];
    \end{verbatim}
   Then we create the problem, giving it the name {\tt "HS65"} --
    \begin{verbatim}
     P=NLCreateProblem("HS65",3);
    \end{verbatim}
   and change the names of the varables (although these are the default names anyway) and
   set the bounds on the variables.
    \begin{verbatim} 
     NLPSetVariableName(P,0,"X1");
     NLPSetSimpleBounds(P,0,-4.5,4.5);
   
     NLPSetVariableName(P,1,"X2");
     NLPSetSimpleBounds(P,1,-4.5,4.5);

     NLPSetVariableName(P,2,"X3");
     NLPSetSimpleBounds(P,2,-5.,5.);
    \end{verbatim}
    Next we specify the objective function (which in this case depends on all three problem variables)
    \begin{verbatim} 
     v[0]=0;v[1]=1;v[2]=2;
     NLPSetObjectiveByString(P,"Obj",3,v,
         "[x1,x2,x3]","(x1-x2)**2+(x1+x2-10)**2/9+(x3-5)**2");
    \end{verbatim}
    and add an inequality constraint
    \begin{verbatim} 
     v[0]=0;v[1]=1;v[2]=2;
     NLPAddInequalityConstraintByString(P,"I1",0.,1.e40,3,v,
         "[x1,x2,x3]","48-x1**2-x2**2-x3**2");
    \end{verbatim}
    And that' all there is to it.

    If instead we had subroutines to evaluate the objective ({\tt o} and {\tt do} and {\tt ddo} to
    evaluate the first and second derivatives) and constraint ({\tt c}, {\tt dc} and {\tt ddc})
    the code would change slightly:
    \begin{verbatim} 
     v[0]=0;v[1]=1;v[2]=2;
     NLPSetObjectiveByString(P,"Obj",3,v,o,do,ddo,NULL,NULL):
     NLPAddInequalityConstraintByString(P,"I1",0.,1.e40,3,v,
         c,dc,ddc,NULL,NULL);
    \end{verbatim}

  \subsection{Using the AddGroup routines}

  This approach uses the same definition as the SIF file for HS65 would.
  The objective consists of three groups with the same group function, 
  none has any nonlinear elements, and the first has no constant part
  to the linear element.

   So we will define one {\tt LNGroup\-Function}, passing it routines which
  square the argument and evaluate the derivatives. We need one 
   {\tt LNElement\-Function}, for the single non-linear element in 
  the constraint. This evaluates the same function as the group function,
  but element functions, unlike groups, which take a scalar argument, take a 
  vector as argument.

  First we include the API prototypes:
\begin{verbatim}
#include <NLPAPI.h>
\end{verbatim}
  Then define two sets of three functions, which will be used for the
  group and element functions and their derivatives.
\begin{verbatim}
double gSq(double x){return(x*x);}
double dgSq(double x){return(2*x);}
double ddgSq(double x){return(2);}

double fSq(int n,double *x){return(x[0]*x[0]);}
double dfSq(int i,int n,double *x){return(2*x[0]);}
double ddfSq(int i,int j,int n,double *x){return(2);}
\end{verbatim}
  The main program and declarations --
\begin{verbatim}
int main(int argc,char *argv[])
 {
  NLProblem P;
  NLGroupFunction g;
  NLElementFunction f;
  NLNonlinearElement ne;
  int group;
  NLVector a;
  double x0[3];
  NLLancelot Lan;
  double x[3];
  int constraint;
  int element;
  int v[1];
  int i;
  int rc;
\end{verbatim}
  We are now ready to create the problem and a group and element function
\begin{verbatim}
  P=NLCreateProblem("HS65",3);
  g=NLCreateGroupFunction(P,"L2",gSq,dgSq,ddgSq,NULL,NULL);
  f=NLCreateElementFunction(P,"fSq",1,NULL,fSq,dfSq,ddfSq,NULL,NULL);

\end{verbatim}
  Note that for {\tt NLCreateElementFunction} the number of internal 
  variables is passed (i.e. the actual number of unknowns used by the
  function), as well as a ``range transformation'', which by default is
  the identity. The last two arguments allow data to be passed to the 
  element function. The second argument is an "element function type",
  and should be unique to the element function.

  The objective function $(x_1-x_2)^2+(x_1+x_2-10)^2+(x_3-5)^2$, consists 
  of three groups, all with the same group function, but different linear
  parts. $(x_1-x_2)^2$:
\begin{verbatim}
  group=NLPAddGroupToObjective(P,"OBJ1","L2");
  rc=NLPSetObjectiveGroupFunction(P,group,g);
  a=NLCreateVector(3);
  rc=NLVSetC(a,0,1.);
  rc=NLVSetC(a,1,-1.);
  rc=NLVSetC(a,2,0.);
  rc=NLPSetObjectiveGroupA(P,group,a);
  NLFreeVector(a);
\end{verbatim}
  $(x_1+x_2-10)^2$:
\begin{verbatim}
  group=NLPAddGroupToObjective(P,"OBJ2","L2");
  rc=NLPSetObjectiveGroupFunction(P,group,g);
  a=NLCreateVector(3);
  rc=NLVSetC(a,0,1.);
  rc=NLVSetC(a,1,1.);
  rc=NLVSetC(a,2,0.);
  rc=NLPSetObjectiveGroupScale(P,group,9.);
  rc=NLPSetObjectiveGroupA(P,group,a);
  rc=NLPSetObjectiveGroupB(P,group,10.);
  NLFreeVector(a);
\end{verbatim}
  $(x_3-5)^2$:
\begin{verbatim}
  group=NLPAddGroupToObjective(P,"OBJ3","L2");
  rc=NLPSetObjectiveGroupFunction(P,group,g);
  a=NLCreateVector(3);
  rc=NLVSetC(a,0,0.);
  rc=NLVSetC(a,1,0.);
  rc=NLVSetC(a,2,1.);
  rc=NLPSetObjectiveGroupA(P,group,a);
  rc=NLPSetObjectiveGroupB(P,group,5.);
  NLFreeVector(a);
\end{verbatim}

  Next come bounds on the variables:
\begin{verbatim}
  rc=NLPSetSimpleBounds(P,0,-4.5,4.5);
  rc=NLPSetSimpleBounds(P,1,-4.5,4.5);
  rc=NLPSetSimpleBounds(P,2,-5.,5.);
\end{verbatim}
 And finally the single nonlinear inequality constraint, which is the 
 trivial group with three nonlinear elements. The default bounds on the
 constraint are 0 on the left, and $\infty$ on the right, so we need not
 change the bounds.
\begin{verbatim}
  constraint=NLPAddNonlinearInequalityConstraint(P,"C1");
  rc=NLPSetInequalityConstraintB(P,constraint,-48.);
  v[0]=0;
  ne=NLCreateNonlinearElement(P,"Sq1",f,v);
  element=NLPAddNonlinearElementToInequalityConstraint
                                   (P,constraint,-1.,ne);
  NLFreeNonlinearElement(P,ne);

  v[0]=1;
  ne=NLCreateNonlinearElement(P,"Sq2",f,v);
  element=NLPAddNonlinearElementToInequalityConstraint
                                   (P,constraint,-1.,ne);
  NLFreeNonlinearElement(P,ne);

  v[0]=2;
  ne=NLCreateNonlinearElement(P,"Sq3",f,v);
  element=NLPAddNonlinearElementToInequalityConstraint
                                   (P,constraint,-1.,ne);
  NLFreeNonlinearElement(P,ne);
\end{verbatim}

  \subsection{Invoking LANCELOT to solve the problem}
  No matter which method was used to define the problem, the invocation of LANCELOT
  (Oh great and powerful LANCELOT, we pray that you find a solution ... ) is the same.
  First we call the constructor for the NLLancelot object, then set the initial guess
  and ask for the minimization to be performed.
\begin{verbatim}
  Lan=NLCreateLancelot();

  x0[0]=-5.;
  x0[1]=5.;
  x0[2]=0.;
  rc=LNMinimize(Lan,P,x0,(double*)NULL,x);
\end{verbatim}
  The rest of the example simply prints the solution 
\begin{verbatim}
  printf("Solution is (");
  for(i=0;i<3;i++)
   {
    if(i>0)printf(",");
    printf("%lf",x[i]);
   }
  printf(")\n");
\end{verbatim}
  and any errors that may have occured. I've embedded a couple, just to
  be tricky. We can test for an error with the return codes, or with the
  {\tt LNGetError} function.
\begin{verbatim}
  printf("There were %d errors\n",NLGetNErrors());
  if(NLError())
   {
    for(i=0;i<NLGetNErrors();i++)
     {
      printf(" %d line %d, file %s, Sev: %d\n",i,
           NLGetErrorLine(i),NLGetErrorFile(i),NLGetErrorSev(i));
      printf("    Routine: \"%s\"\n",NLGetErrorRoutine(i));
      printf("    Msg: \"%s\"\n",NLGetErrorMsg(i));
     }
   }
\end{verbatim}
  And the final step, which really should be done, is to return all the
  memory used to the system.
\begin{verbatim}
  NLClearErrors();
  NLFreeGroupFunction(g);
  NLFreeElementFunction(f);
  NLFreeLancelot(Lan);
  NLFreeProblem(P);
  return(0);
 }
\end{verbatim}

The program produced the following output --
\begin{verbatim}
 objective function value =   9.53528856489003D-01

            X0000001          3.65046233137863D+00
            X0000002          3.65046219722889D+00
            X0000003          4.62041670283010D+00
            C1                0.00000000000000D+00
Solution is (3.650460,3.650460,4.620420)
There were 0 errors
\end{verbatim}
\end{document}
